%%%%%%%%%%%%%%%%%%%%%%%%%%
%%%%%   INTRODUCTION %%%%%
%%%%%%%%%%%%%%%%%%%%%%%%%%
\chapter{Introduction}
\label{ch:1.0}

%This chapter describes the fundamental concepts used throughout the thesis.
%Section \ref{seq:malaria.intro} introduces the theme of malaria as a global disease.
%A brief description of malaria and its burden in human populations, focusing the current panorama for malaria epidemiology in tropical and subtropical countries.
%Epidemiological tools used to estimate malaria transmission intensity are described in Section \ref{seq:estimating}, distinguishing between the currently most used approaches and explaining why serology could be an alternative.
%An example of an important serological study is shown in Section \ref{seq:example}, where sero-epidemiology was used beyond the classical tools in order to estimate malaria transmission intensity with positive results.
%Section \ref{seq:objectives} describes the outline structure of the present thesis, explaining the objectives that led to its development at the London School of Hygiene and Tropical Medicine.
%A brief description of its development alongside the human populations and a description of current panorama for malaria epidemiology in tropical and subtropical countries are made in Section \ref{seq:malaria.intro}.
%is then introduced, where recent (and consecutive) interventions in these sites continue to decrease the risk of contracting malaria.
%An example of application using sero-epidemiological methods is given in Section \ref{seq:example} with the present challenges briefly described in Section \ref{seq:challenges}.
%Section \ref{seq:objectives} makes a short description of the outline of the present thesis, as well as the fundamental objectives that led this project to gain shape.
%adjusts as the `classical' tools developed to measure situations of high malaria transmission intensity may present limitations when it comes to measure low endemic malaria transmission intensity.
%Malaria is a severe public health problem and remains the leading cause of death in developing countries.

Malaria is a parasitic disease described since ancient times, and until this day it continues to be a major health problem.
With approximately 400 million people world-wide currently infected and estimates for 130 million new cases every year \cite{perlmann2002malaria}, malaria is one of the principal causes of death and disease caused by a parasite \cite{vitoria2009global}.
%continuously being a leading cause of death and disease across many tropical and subtropical countries, going as far as exerting selective pressures, such is the example of the sickle cell trait as a protective adaptation \cite{}.
The term malaria originates from the 18th century italian expression \textit{mala aria}, meaning `bad air', referring the foul air evaporating\textcolor{red}{/miasma} found in stagnant waters of marshes that used to be thought as the origin of the disease \cite{warrell2002essential}.
The real causative agent of malaria was only discovered in 1897, after Ronald Ross identified parasites in a mosquito that had previously fed on an infected patient \cite{ross1897observations}.
The understanding of these parasites' life cycle of development and transmission was the beginning for the creation of specialised and more focused methods for malaria treatment and control.
%This first understatement of the life cycle of development of the causative agent of malaria in humans and a vector -- the mosquito belonging to the genus \textit{Anopheles} -- led to the foundation for methods of combating the disease.

%The main vectors for malaria transmission in
The main carriers of malaria parasites known to affect humans are some species and subspecies of mosquitoes belonging to the genus \textit{Anopheles}.
More precisely, the female mosquitoes, as they must take blood meals to support the development of successive batches of eggs \cite{who2017framework}.
%When infected, by biting exposed individuals it transmits the protozoan parasites of the genus \textit{Plasmodium} into the human hosts' blood steam, originating infections 9 to 10 days after \cite{who2017framework}.
%This alternation between humans and mosquito host represents the biological cycle of transmission of the malaria parasite.
If infected, the mosquito transmits protozoan parasites of the genus \textit{Plasmodium} into the human hosts' blood stream, where they travel to the liver to multiply.
After five to fifteen days without apparent symptoms, the matured parasites re-emerge into the blood stream, targeting and invading the red blood cells \cite{who2017framework}.
By the time an infected individual shows the primary symptoms of malaria (chills and fever), the parasites have already multiplied immensely, clogging blood vessels and rupturing blood cells \cite{warrell2002essential}.



%%%%%%%%%%%%%%%%%%%%%%%%%%
% MALARIA & EPIDEMIOLOGY %
%%%%%%%%%%%%%%%%%%%%%%%%%%
%\section{Malaria epidemiology}

\section{Epidemiology of the burden of human malaria}
\label{seq:malaria.intro}
There are four distinct \textit{Plasmodium} parasite species known to infect humans: \textit{P. falciparum}, \textit{P. vivax}, \textit{P. malariae}, and \textit{P. ovale} \cite{who2017world}.
%Of all the human malaria parasites, \textit{P. falciparum} is the most virulent, being responsible for the majority of malaria related morbidity and mortality, accounting for 99\% of the estimated malaria cases in the sub-Saharan Africa in 2016 \cite{who2017world}.
Of all, \textit{P. falciparum} is the most virulent, being the principal cause of malaria morbidity and mortality \cite{perlmann2002malaria}.
%and the inferences made throughout this thesis deal exclusively with this pathogen.
An individual infected with this parasitic species usually has prolonged fevers, with the massive number of destroyed red blood cells forming clots that can block the blood vessels in vital organs \cite{perlmann2002malaria,warrell2002essential}.
\textit{P. falciparum} malaria endemicity varies geographically.
The differences in stability of transmission intensity depend on various environmental, demographical, and biological characteristics of \textcolor{red}{quebrar mais as frases para ser mais específico}its vectors, with implications on the human hosts \cite{carter2002evolutionary, snow2002consequences}.
Its spread can be delimited by climatic determinants such as temperature, altitude, rainfall patterns, or humidity, that govern the activity and abundance of the \textit{Anopheles} mosquitoes \cite{warrell2002essential}.
Man made environmental changes like deforestation, extensive irrigation systems, or water dams can also cause transmission intensity to change.
The combination of these aspects makes \textit{P. falciparum} endemicity usually found in tropical, subtropical and temperate regions like the sub-Saharan Africa \cite{warrell2002essential}, where it accounts for 99\% of the estimated malaria cases in 2016 \cite{who2017world}.

\textcolor{red}{On these tropical ...} On these locations, populations are continuously exposed to a pattern of somewhat constant rate of malaria infections.
%Being vector dependent, the action of this parasite is refrained by several determinants that influence the , such as the temperature, rainfall and humidity, altitude, malaria seasonality, and even man-made events such as deforestation, the use of certain crops, or urbanisation.
%With usually found in tropical, subtropical and temperate regions, accounting for 99\% of the estimated malaria cases in the sub-Saharan Africa in 2016 \cite{who2017world}.
%In these regions, malaria is still the leading cause of morbidity and mortality in children younger than 5 years old and pregnant women \cite{who2017world}.
%Malaria is indigenous to 91 countries, with an estimated 216 million cases and 445,000 deaths worldwide.
%Around 90\% of these cases and 91\% of all malaria deaths occur in sub-Saharan Africa, where malaria is the leading cause of morbidity and mortality in children younger than 5 years old and pregnant women \cite{who2017world}.
%In such situations of stable endemic \textit{P. falciparum} parasites,
During the course of natural infections, individuals develop specific antibodies against the malarial antigens.
With multiple episodes of infection over time, a protective immunity will gradually build up, reducing manifestations of severe disease \cite{perlmann2002malaria}.
Because this process of achieving effective protection takes time, the antimalarial immunity in malaria endemic countries is said to be `age-dependent' \cite{ondigo2014estimation}, justifying the reason for the primary risk group being children younger than 5 years old, who have yet to develop an efficient immune system.
Pregnant woman are also a vulnerable group to malaria infection, as pregnancy reduces immunity \textcolor{red}{afecta o sistema imunitário e as defesas para combater as doenças} against the disease \cite{who2017world, carter2002evolutionary}.

Contrarily to some diseases that cause long-lasting immunity \textcolor{red}{dar exemplos ou reescrever pra ser mais ligeiro}, i.e., the produced antibodies endure throughout an individual's life, the developed antimalarial antibodies wane over time.
After a prolonged interval without reinfection, immunologically protected individuals can revert to an infant-like status and once again become vulnerable to contract a renewed infection.
%These specific protective antibodies are thought to have some effect on the transmission dynamics of malaria . Loss of protective immunity

The last decade has witnessed a rise in commitment to malaria control in African malaria endemic countries \cite{kitua2011conquering}.
Effective actions in heavy malaria burdened sites have shown success in reducing the parasite's registered morbidity and mortality.
Fundings for disease control and prevention, although reportedly still less than required \cite{pigott2012funding}, allowed for campaigns to include insecticide-treated mosquito nets, insecticide spraying actions, and facilitation of access to curative and preventive antimalarial drugs for millions of people in risk \cite{who2017world}.

\textcolor{red}{O que falta: conecção com a secção seguinte... Temos o controlo, tem vindo a diminuir e essa transmissão no fundo -- com a diminuição temos de estimar essa malária}

%%%%%%%%%%%%%%%%%%%%%%%%%%%%%%%%%%%%%%%%%%%%%
% ESTIMATING MALARIA INTENSITY TRANSMISSION %
%%%%%%%%%%%%%%%%%%%%%%%%%%%%%%%%%%%%%%%%%%%%%
\section{Estimating malaria transmission intensity}
\label{seq:estimating}


%%%%%%%%%%%%%%%%%%%%%%%%%%%%%%%%%%%%%%%%%%%%%
\subsection{Conventional measures to estimate transmission intensity of infection}

Campaigns for control and elimination require knowledge and stratification of malaria transmission intensity amongst the intervened populations \cite{who2017framework, who2015guidelines}.
Based on the parasites' life cycle and its influencing determinants, several approaches can be used to estimate such transmission rate.
Currently, the most used measures for malaria transmission focus \textcolor{red}{on counting the numbers of detecteable...} their prevalence analyses on clinically detectable cases of infection.
%For this thesis, the most known measures are grouped as the `classical approaches'.
%These measures usually focus on clinically detectable cases of infection, presenting similar strengths and limitations.
%For this thesis, measures used are grouped as two distinct groups.
%The first one, the `classical' approaches, identifies the most used measures.
%Those that focus on clinically detectable cases of infection.
These measures are usually based on active or passive case detection \cite{who2017framework,doolan2002malaria} \textcolor{red}{definir conceitos não depois!!}.
Infected individuals are identified by active searches for diseased patients, or those symptomatic patients come into specialised health facilities seeking care for their illness.
In both situations, analyses are performed %mostly 
on individuals presenting symptoms that evidence a possible case of infection \cite{who2017framework}.
Measures such as the parasite rate (PR), also known as prevalence of infection, for assessing the proportion of individuals \textcolor{red}{(prevalence)} with blood-stage infections amongst a community, or the spleen rate, for identifying the prevalence of palpable enlarged spleens due to malaria infection (a effect more commonly observed in first time infected individuals), are examples of possible symptomatic population focused approaches.
% Both methods usually assume prevalence of infection as an annual rate.
%of inference for transmission intensity uniquely based on population screening.
In order to be effective, these measures depend greatly on diagnosis %\textcolor{red}{and treatment} 
given by established services from the public, or private health sectors, as well as coordinated community services, that are the first action for symptom assessment and treatment, reporting the cases to health facilities \cite{who2017framework}.
Alternatively, estimation of transmission intensity can be done by studies measuring the density of \textit{Anopheles} mosquitoes near the inspected populations.
The proportion of infected mosquitoes in a region directly correlates to the capacity of these carriers to transmit malaria within that area.
This insect proportion also reflects the number of infected, and potentially infectious, human individuals \cite{who2015guidelines}.
The entomological inoculation rate (EIR) measures the number of infective mosquito bites received per person, in a population, over a defined period of time \cite{who2017framework}.
%by calculating the proportion of mosquitoes and their human biting rate \cite{warrell2002essential}, 

As interventions' coverage \textcolor{red}{definir coverage!} increase, heterogeneity in transmission intensity across different regions is likely to occur.
When assessing transmission rates across different sites, reports given by measures such as PR or EIR allow to identify determinant variables related to the parasites, the carriers, or even the human hosts \cite{who2015guidelines}.
The analysis of these potential risk variables sided with the incidence patterns they cause can be used to define high transmission areas
% , or even identify hotspots for transmission
, and act accordingly \cite{who2015guidelines,world2015global}.

%Although practical and broadly used, by often depending on clinical cases, these measures can be imprecise and fail if there is a large prevalence of asymptomatic infections \cite{warrell2002essential}.
Although practical and broadly used, these measures present limitations.
In low transmission settings, the difficulty to estimate malaria incidence intensifies.
Low transmission environments are characterised by high percentage of non symptomatic cases of infection, and residual number of detectable infected mosquito bites.
Being asymptomatic or undiagnosed, the individuals will remain invisible to the health system while still contributing to the cycle of malaria transmission \cite{world2015global}.
Sites affected by seasonality that regularly shift between extreme high and low transmissions intensities also present a challenge to obtain accurate results \cite{cameron2015defining,o2007parasite}.
Regions where malaria prevalence has been efficaciously reduced, or have recently been focused by campaigns, still need to be monitored in order to change interventions from malaria control and elimination, to disease eradication.
For sites where malaria incidence is currently low, alternative approaches can be used to estimate the transmission rates \cite{corran2007serology}. 
Serological antibody based approaches can be used, as follows.
%If the study site is affected by seasonality, the results produced can also be misleading \cite{cameron2015defining}.
%The approaches present limited sensitivity to extreme high or low transmission intensities \cite{o2007parasite}.
%Isto acontece porque torna-se difícil associar os simptomas unicamente à malaria!
%In some settings the density of parasitaemia is so low in a substantial proportion of individuals that it cannot be detected with current routine diagnostic tools. These people unwittingly contribute to the cycle of malaria transmission. If future disease control and elimination strategies are to succeed, they will need to take into account this large “infectious parasite reservoir”. The expected development and availability over the next decade of new tools and approaches should help the detection and targeting of this reservoir and the clearing of plasmodia from asymptomatic carriers \cite{world2015global}.
%\textcolor{red}{geographical and demographical} 
%To deal with the  key approach to optimizing malaria responses within a country will be structuring programmes in response to stratification by malaria burden and based on an analysis of past malaria incidence data, risk determinants related to the human host, parasites, vectors and the environment that together with an analysis of access to services..
%The standard approaches are useful to estimate presence of infection in situations where endemic malaria occurs at high intensity rates, with several symptomatic cases at a time.
%It is then necessary to study possible alternatives \cite{malera2011research} in order to optimise population screenings \cite{sachs2002economic, stewart2009rapid}.
%as they tend to be expensive, time-consuming, and even with some lack of precision \cite{sachs2002economic, stewart2009rapid}.
%Assessing malaria transmission intensity and evaluating the impact of interventions is complicated in areas where transmission has been substantially reduced, as low endemic malaria often does not show evident symptoms \cite{}.
%Therefore, alternative approaches are required to assess malaria transmission and evaluate intervention programmes.


%%%%%%%%%%%%%%%%%%%%%%%%%%%%%%%%%%%%%%%%%%%%%
\subsection{Serology as an epidemiological tool}

\textcolor{red}{O que é serologia, uma serologia approach

Posso meter um parágrafo à parte (em baixo) a definir os anticorpos.}
Antibody are protective proteins produced by the immune system.
The antibodies are able to recognise and target particular foreign substances, the antigen molecules.
Using serological methods, malaria transmission intensity is assessed by measuring the levels of specific anti-malarial antibodies produced upon exposure to the parasites, and by identifying antigens present in the serum \cite{corran2007serology, drakeley2005estimating}.
Serology allows then to estimate the population level of disease transmission by appraising how a population boosts its immunity, as a response to the presence or absence of infection.
%Since any healthy individual in expected to develop protective antibodies, t
This approach shifts the focus away from an outcome merely based on infection.
%For that situation, a second group identifies serology as a way to measure malaria transmission intensity in situations of low and endemic malaria transmission.
%When symptomatic malaria infections occur frequently the `classical' approaches are the most commonly used measures.
%With the reduction of malaria symptomatic cases on various sites across Africa \cite{}
%Classic deterministic models
%The improvement over the last decade resulted in a 
%led to a change in malaria intervention objectives, shifting from controlling towards elimination \cite{kitua2011conquering}.
%Serology is based on the human anti-malarial antibodies gain by individuals exposed to malaria parasites.
\textcolor{red}{In normal conditions}, an exposed healthy individual will gradually develop and accumulate the anti-malarial antibodies.
This effect can be better seen in older children and adults, where the frequency and severity of the disease are reduced when compared to younger individuals \cite{snow2002consequences}.
Being a direct response to malaria, the presence of \textit{P. falciparum}-specific antibodies reflects the cumulative (age-dependent) exposure to multiple infections over time \cite{van2015serology}.
%These antibodies can be detected and measured in the serum.

\textcolor{red}{Blood samples are usually taken, producing... SER MAIS PRECISO} Serological samples taken at a certain time point can provide information about whether or not the individual has been infected before that time point \cite{hens2012modeling}.
This ability allows serology to function as a proxy measure of historical malaria transmission, even in low transmission settings.
Serology has increasingly been incorporated in cross-sectional and longitudinal studies to monitor recent population changes in transmission intensity \cite{cook2010using, cook2011serological, hay2008measuring}, identify hotspots for transmission, and evaluate effectiveness of malaria eradication efforts \cite{bruce1973seroepidemiological}.

Based on this approach, data sets can be based on random samples drawn from the population and still detect possible disease transmission heterogeneity across different epidemiological situations.
The differentiation across multiple sites provides a better source of information than active or passive case detection that usually inspect only those who appear suspected of being infected, with possible biased results or inaccurate representative cases \cite{nkumama2017changes}.
Under the assumptions of acquired immunity (i.e. gradual learning of the immune system upon multiple exposures) and antibody persistence over a long period time, data from a single cross-sectional survey can be used to generate a point estimate of the current disease transmission intensity.
The measure can also analyse potential historical changes in transmission intensity that led to a variation in exposure to the infection \cite{hens2012modeling}.
%Because serological markers provide information on cumulative exposure over time they are particularly well suited for evaluating long-term transmission trends \cite{corran2007serology,drakeley2005estimating}. Data from a single cross-sectional serological survey can be used
%With the recent progress made in reducing the global malaria burden, alternative approaches that correctly identify asymptotic infections are becoming fundamental \cite{malera2011research}.
%As a proxy measure of malaria transmission, serological responses to P. falciparum antigens have shown a robust and consistent correlation with estimates of EIR \cite{corran2007serology}, and thus 
%Immunity in malaria does not fully protect against infection of disease and it may have poor immune memory \cite{struik2004does}.
%Thus, herd immunity does not tend to develop.
%In highly endemic areas, immunity reduces parasite densities in older children and adults and it reduces the frequency and severity of the disease 
%challenges for malaria em low transmission settings \cite{stresman2012malaria} principal serologic markers to detect malaria em low transmission settings \cite{bousema2010serologic} (pfMSP-1 and pfAMA-1)
%Under stable endemic malaria conditions, variation in transmission can continue even with very few vectors.
%High levels of immunity develop within the population due to regular and often continuous transmission \cite{warrell2002essential}.
%Effective antiparasitic immunity is achieved only after multiple infections
%The advantage of serological data in quantifying parasite exposure instead of infection has recently brought interest in sero-epidemiological studies of malaria \cite{}.
%Alternative method is to examine the prevalence of some markers of previous infection that is present in blood serum -- a measure of the proportion of humans currently with antibodies that developed in response to malaria infection \cite{corran2007serology}.
%A unique attribute of antibody measurements is that they provide an immunological record of an individual's exposure or vaccination history, and thus integrate information over time \cite{corran2007serology}.
%Typically seroprevalence rises with age and gives a robust measure of previous infection compared with age.
%Introduction to serology
%Individuals are born seronegative but can be converted into seropositive upon malaria exposure. In the absence of continuous frequent malaria exposure, these individuals can then revert to a seronegative immunological state.
%Explain about the relation between classical EIR and the measure of transmission intensity on serology (SCR).
%Explicar o processo de 'ser exposto' e passar ao estado infectado, e depois recuperar. Falar sobre o processo de desenvolver anticorpos específicos no processo de combater a infeção.
%\cite{arnold2017measuring}

%\textbf{In areas where malaria is highly seasonal, or remains within low levels, asymptomatic cases are a significant portion of the populations \cite{harris2010large}.}



%%%%%%%%%%%%%%%%%%%
% BENCHMARK STUDY %
%%%%%%%%%%%%%%%%%%%
\section{Northeast Tanzania as a serological benchmark study}
\label{seq:example}

%Serology has been gaining importance as it has proven to be effective when measuring transmission intensity in situations of low and endemic malaria transmission.
Sero-epidemiology has already been assessed as a good alternative to analyse situations of malaria in states of pre-elimination and elimination \cite{corran2007serology}.
A benchmark example is the study from where the data set used throughout this thesis originates \cite{drakeley2005altitude}.
Using serology based methods the study in Northeast Tanzania was able to describe the effects of altitude and estimated rainfall across different sites with varying intensities of transmission.
By measuring the seroprevalence levels of individuals in different villages the project measured the heterogeneity of \textit{P.falciparum} malaria transmission intensity, confirming both variables to have a measurable impact on the disease force of infection.
\\
Following the described article, several published articles used and improved the sero-epidemiological inferences.
Based on the same data set, studies of methods and approaches in various research fields were developed.
Are examples serological analyses inquiring about the trends in malaria endemicity \cite{drakeley2005estimating}, genetic studies on populations exposed to \textit{P. falciparum} parasite \cite{enevold2007associations, sepulveda2017malaria}, and development of specific mathematical model, used for serological analyses \cite{bosomprah2014mathematical}.
%Due to the variety and quality of information gathered, this data set has been used on multiple different studies with the focus being serological analysis, inquiring about the trends in malaria endemicity \cite{drakeley2005estimating}; genetic studies on populations exposed to \textit{P. falciparum} parasite \cite{enevold2007associations,sepulveda2017malaria}; and development of specific mathematical model, used for serological analysis \cite{bosomprah2014mathematical}.



%%%%%%%%%%%%%%%%%%%%%%
% CURRENT CHALLENGES %
%%%%%%%%%%%%%%%%%%%%%%
\section{Current challenges on malaria epidemiology}
\label{seq:challenges}

The multidisciplinary investment to control and aid populations hurt by the endemic malaria burden is visible \cite{who2017world}.
Nowadays, severe malaria develops only in a minority of sites, as effective campaigns have been able to control and reduce disease transmission intensity substantially \cite{marsh1995indicators}.
Low transmission settings are now registered across various regions \cite{cook2010using}.
%More regions are reducing malaria to low transmission settings with campaigns, switching their focus from sustained control to elimination \cite{cook2010using}.
%All efforts have resulted in areas of low transmission intensity,
%or areas where transmission has been reduced substantially, 
%malaria control programmes can start considering switching from sustained control initiatives to elimination \cite{cook2010using}.
%The  measure the fraction of the population with particular conditions at some point in time.
All measures here presented estimate malaria transmission intensity on the human population.
However, none of them is a perfect indicator.
%Treated infections can clear rapidly and even untreated infections can also clear after some time, so the classical approaches are but measures of recent infection.
Prevalence of malaria presents a pattern that increases with age in young children and then declines throughout adolescence and adulthood.
This age-dependent `peak-shift' can be difficult to estimate, making approaches such as PR or EIR poor indexes of transmission intensity over time.
These measures can be used as good alternatives to estimate recent infections.
%Clinical incidence also displays age-specific patterns that can differ by endemicity, making the measures unreliable indexes of transmission intensity, as the screened regions may present heterogeneity in malaria transmission intensity.
For serological analyses, some infections may be treated and clear before an immune response develops.
In other scenarios, individual characteristics, such as genetics, can also present a significant effect in the immune response.
This possible lack of immunity development, as well as waning immune responses, can affect the accuracy of seroprevalence.



%%%%%%%%%%%%%%%%%%%%%%%%
% OBJECTIVES & OUTLINE %
%%%%%%%%%%%%%%%%%%%%%%%%
\section{Objectives and outline}
\label{seq:objectives}

\textcolor{red}{analisar os dados ... in order to estimate}
The main objective of the thesis is to estimate malaria transmission intensity across different defined sites from the Northeast Tanzania.
Using infection and serological \textcolor{red}{surveys/samples} from three different \textit{P. falciparum} antigens -- merozoite surface protein 1 (MSP1$_{19}$), merozoite surface protein 2 (MSP2), and apical membrane antigen 1 (AMA1) --, one expects to identify the principal risk factors influencing malaria transmission, as well as measure the incidence heterogeneity from different villages with identifiable characteristics.
%identify the expected heterogeneity amongst different sites through the use of different statistical approaches that relate to a real life scenario.
Based on the aphorism that all models are wrong but some are useful \cite{box2005statistics}, different statistical models are fit to the data.
Firstly, by making use of the the more commonly used statistical approaches, to study the detected infection cases.
Afterwards, applying specific serological models to the different known antigens.
%Both approaches aim to describe the disease heterogeneity present across all populations studied.
%both introduced approaches, with aim to describe the disease heterogeneity present across all populations studied.
%The prevalence and seroprevalence values are used, measuring current malaria transmission intensity based on past exposure or past interventions.
%Estimate and measure malaria transmission intensity by

As a thesis in Biostatistics, this project was structured in a way that would focus different academic objectives, while maintaining a coherent -- and hopefully interesting -- line of thought.
With its fundaments in the matters of malaria and malaria sero-epidemiology, this project grants the opportunity to:
\begin{itemize}
\item Work with well recorded cross-sectional data, used in renowned sero-epidemiological studies;
\item Build different generalised linear models to characterise the exposure factors and study effects causing heterogeneity in transmission intensity levels;
\item Make use of the specific reverse catalytic models to corroborate the previous heterogeneity inferences, and create serological profiles for different villages based on different biological and epidemiological assumptions;
\item Describe and propose an innovative extension of the more broadly used reverse catalytic models;
\item Study the implications of applying different models to the same data set by changing between strategies.
\end{itemize}
%When performing inference for serology, two of the three models here described are known approaches, with one of the models here proposed being innovative.
%\textbf{
%objective: ja está feito: SCR constante ao longo do tempo\\ changes in SCR and changes in SRR e ver se as condições mudam com o que já está reportado.}
%The main objective of the thesis is to estimate and measure malaria transmission intensity from a set of different groups of villages, identify the expected heterogeneity amongst different sites through the use of different statistical approaches that relate to a real life scenario.
%Based on the aphorism that all models are wrong but some are useful \cite{}, different models are created and compared.
%The prevalence and seroprevalence values are used, measuring current malaria transmission intensity based on past exposure or past interventions.
%The thinking that was made during the creation of the project.\\
%The analysis that were made, the models created and the order in which they were created. The creation of the so called Infection Model.\\
%Etc.
%George P. Box -- \emph{´All models are wrong, but some are useful.'} -- The practical question is to how wrong do they have to be to no be useful?
%With this thesis different statistical approaches are described and used onto the same data serological set.
%All different models created aim to describe the disease heterogeneity, present across all populations studied.

The following chapters will specify the situation of malaria in the two studied regions of the Northeast Tanzania, as well as introduce, and describe the collected data (Chapter \ref{ch:2.0}).
The statistical theory used is described further ahead (Chapter \ref{ch:3.0}).
The models' analyses and inferences are then presented, firstly focusing the infection status of each individual as the outcome of interest.
Afterwards, by applying the alternative statistical methodologies onto the serological outcomes (Chapters \ref{ch:4.0} and \ref{ch:5.0}, respectively).
Lastly, the different methodologies and results are discussed, attending the statistical and epidemiological backgrounds of this project (Chapter \ref{ch:discussion}).