%%%%%%%%%%%%%%%%%%%%%%%%%%
%%%%%   INTRODUCTION %%%%%
%%%%%%%%%%%%%%%%%%%%%%%%%%
\chapter{Introduction}
\label{ch:1.0}

% % \usepackage{multirow}
% % \usepackage{booktabs}
% \begin{table}
% \centering
% \begin{tabular}{llccc} 
% \toprule
% \multicolumn{1}{c}{\multirow{2}{*}{Transect}} & \multicolumn{1}{c}{\multirow{2}{*}{Village}} & \multicolumn{3}{c}{Selected RCM}    \\ 
% \cmidrule{3-5}
% \multicolumn{1}{c}{}                          & \multicolumn{1}{c}{}                         & MSP1       & MSP2     & AMA1        \\ 
% \midrule
% Rombo                                         & Mokala                                       & M$_{0}$    & M$_{0}$  & M$_{0}$     \\
%                                               & Machame Aleni                                & M$_{0}$    & M$_{0}$  & M$_{1,2}$   \\
%                                               & Ikuini                                       & M$_{0}$    & M$_{0}$  & M$_{0}$     \\
%                                               & Kileo                                        & M$_{0}$    & M$_{0}$  & M$_{0}$     \\ 
% \cmidrule{2-5}
% N. Pare                                       & Kilomeni                                     & M$_{0}$    & M$_{0}$  & M$_{0}$     \\
%                                               & Lambo                                        & M$_{0}$    & M$_{0}$  & M$_{0}$     \\
%                                               & Ngulu                                        & M$_{0}$    & M$_{0}$  & M$_{0}$     \\
%                                               & Kambi ya Simba                               & M$_{0}$    & M$_{0}$  & M$_{0}$     \\ 
% \cmidrule{2-5}
% S. Pare                                       & Bwambo                                       & M$_{0}$    & M$_{2}$  & M$_{0}$     \\
%                                               & Mpinji                                       & M$_{1,2}$  & M$_{2}$  & M$_{0}$     \\
%                                               & Goha                                         & M$_{0}$    & M$_{0}$  & M$_{1,2}$   \\
%                                               & Kadando                                      & M$_{0}$    & M$_{0}$  & M$_{0}$     \\ 
% \cmidrule{2-5}
% W. Usamb. 1                                   & Emmao                                        & M$_{0}$    & M$_{0}$  & M$_{0}$     \\
%                                               & Handei                                       & M$_{0}$    & M$_{0}$  & M$_{0}$     \\
%                                               & Tewe                                         & M$_{0}$    & M$_{0}$  & M$_{0}$     \\
%                                               & Mn'galo                                      & M$_{0}$    & M$_{0}$  & M$_{0}$     \\ 
% \cmidrule{2-5}
% W. Usamb. 2                                   & Kwadoe                                       & M$_{2}$    & M$_{0}$  & M$_{0}$     \\
%                                               & Funta                                        & M$_{0}$    & M$_{2}$  & M$_{0}$     \\
%                                               & Tamota                                       & M$_{2}$    & M$_{2}$  & M$_{0}$     \\
%                                               & Mgila                                        & M$_{2}$    & M$_{0}$  & M$_{0}$     \\ 
% \cmidrule{2-5}
% W. Usamb. 3                                   & Mgome                                        & M$_{2}$    & M$_{0}$  & M$_{0}$     \\
% \bottomrule
% \end{tabular}
% \end{table}

%This chapter describes the fundamental concepts used throughout the thesis.
%Section \ref{seq:malaria.intro} introduces the theme of malaria as a global disease.
%A brief description of malaria and its burden in human populations, focusing the current panorama for malaria epidemiology in tropical and subtropical countries.
%Epidemiological tools used to estimate malaria transmission intensity are described in Section \ref{seq:estimating}, distinguishing between the currently most used approaches and explaining why serology could be an alternative.
%An example of an important serological study is shown in Section \ref{seq:example}, where sero-epidemiology was used beyond the classical tools in order to estimate malaria transmission intensity with positive results.
%Section \ref{seq:objectives} describes the outline structure of the present thesis, explaining the objectives that led to its development at the London School of Hygiene and Tropical Medicine.
%A brief description of its development alongside the human populations and a description of current panorama for malaria epidemiology in tropical and subtropical countries are made in Section \ref{seq:malaria.intro}.
%is then introduced, where recent (and consecutive) interventions in these sites continue to decrease the risk of contracting malaria.
%An example of application using sero-epidemiological methods is given in Section \ref{seq:example} with the present challenges briefly described in Section \ref{seq:challenges}.
%Section \ref{seq:objectives} makes a short description of the outline of the present thesis, as well as the fundamental objectives that led this project to gain shape.
%adjusts as the `classical' tools developed to measure situations of high malaria transmission intensity may present limitations when it comes to measure low endemic malaria transmission intensity.
%Malaria is a severe public health problem and remains the leading cause of death in developing countries.

Malaria is a parasitic disease described since ancient times, and until this day it continues to be a major health problem.
% With approximately 400 million people world-wide currently infected and estimates
Despite continuous worldwide efforts and investments, malaria is still the principal cause of death and disease caused by a parasite \cite{vitoria2009global}, with estimates for 216 million new infection cases and 445 000 deaths globally in 2016 \cite{who2017world}.
%continuously being a leading cause of death and disease across many tropical and subtropical countries, going as far as exerting selective pressures, such is the example of the sickle cell trait as a protective adaptation \cite{}.
The term malaria originates from the 18th century italian expression \textit{mala aria}, meaning `bad air', referring the foul air evaporating from stagnant waters of marshes that used to be thought as the origin of the disease \cite{warrell2002essential}.
The real causative agent of malaria was only discovered in 1897, after Ronald Ross identified parasites in a mosquito that had previously fed on an infected patient \cite{ross1897observations}.
This understanding of the parasites' life cycle of development and transmission laid the foundations for specialised and more focused methods for malaria treatment and control.
%This first understatement of the life cycle of development of the causative agent of malaria in humans and a vector -- the mosquito belonging to the genus \textit{Anopheles} -- led to the foundation for methods of combating the disease.

%The main vectors for malaria transmission in
The main carriers of malaria parasites known to affect humans are some species and subspecies of mosquitoes belonging to the genus \textit{Anopheles}.
More precisely the female mosquitoes, as they must take blood meals to support the development of successive batches of eggs \cite{who2017framework}.
%When infected, by biting exposed individuals it transmits the protozoan parasites of the genus \textit{Plasmodium} into the human hosts' blood steam, originating infections 9 to 10 days after \cite{who2017framework}.
%This alternation between humans and mosquito host represents the biological cycle of transmission of the malaria parasite.
Usually biting between sunset and sunrise, the infected mosquito transmits parasites of the genus \textit{Plasmodium} into the human hosts' blood stream, where they travel to the liver to multiply.
After five to fifteen days without apparent symptoms, the matured parasites re-emerge into the blood stream, targeting and invading the red blood cells \cite{who2017framework}.
By the time an infected individual might show the primary symptoms of malaria (such as chills, fever, abdominal discomfort, or muscle and joints aches), the parasites have already multiplied immensely, clogging blood vessels and rupturing blood cells \cite{warrell2002essential}.
Later stages of severe malaria usually cause anaemia, hypoglycaemia, acute renal failure, or coma (cerebral malaria), among other symptoms \cite{who2015guidelines}.
% Though\textcolor{red}{Dizer que pode haver indivíduos asimptomáticos}
However, despite the infection, some individuals might remain asymptomatic, not suggesting a case for malaria infection.
Nonetheless, if left untreated, severe malaria could be considered fatal in most cases \cite{who2015guidelines}.



%%%%%%%%%%%%%%%%%%%%%%%%%%
% MALARIA & EPIDEMIOLOGY %
%%%%%%%%%%%%%%%%%%%%%%%%%%
%\section{Malaria epidemiology}

\section{Epidemiology of the burden of human malaria}
\label{seq:malaria.intro}
There are four distinct \textit{Plasmodium} parasite species known to infect humans: \textit{P. falciparum}, \textit{P. vivax}, \textit{P. malariae}, and \textit{P. ovale} \cite{who2017world}.
%Of all the human malaria parasites, \textit{P. falciparum} is the most virulent, being responsible for the majority of malaria related morbidity and mortality, accounting for 99\% of the estimated malaria cases in the sub-Saharan Africa in 2016 \cite{who2017world}.
Of all, \textit{P. falciparum} is the most prevalent, being the principal cause of malaria morbidity and mortality \cite{perlmann2002malaria}.
%and the inferences made throughout this thesis deal exclusively with this pathogen.
% An individual infected with this parasites usually has prolonged fevers, with the massive number of destroyed red blood cells forming clots that can block the blood vessels in vital organs \cite{perlmann2002malaria,warrell2002essential}.
Malaria endemicity varies geographically, from Africa to South east Asia to South America \cite{who2017world}.
The differences in stability of transmission intensity depend on various environmental and demographical characteristics.
Biological traits and preferences from the \textit{Anopheles} mosquitoes are also an influence with implications on human hosts \cite{carter2002evolutionary, snow2002consequences}.
Their spread can be delimited by climatic determinants such as temperature, altitude, rainfall patterns, or humidity, influencing the mosquitoes' activity and abundance \cite{warrell2002essential}.
Man made environmental changes like deforestation, extensive irrigation systems, or water dams can also cause transmission intensity to change.
The combination of these aspects makes \textit{P. falciparum} endemicity usually found in tropical, subtropical, and temperate regions like the sub-Saharan Africa \cite{warrell2002essential}, where it accounts for 99\% of the estimated malaria cases in 2016 \cite{who2017world, who2017framework}.
% This parasitic species will be the one focused throughout the thesis.

% parágrafo dos Ab e 'age-dependent' immunity passou para a serologia!!

The last decade has witnessed a rise in commitment to malaria control in endemic countries \cite{kitua2011conquering}.
Effective actions in heavy malaria burdened sites have shown success in reducing the parasite's registered morbidity and mortality.
% \textcolor{red}{Insightful studies followed by effective actions in heavy malaria burdened sites have shown success in reducing the parasite's registered morbidity and mortality. ONDE ESTÃO OS ESTUDOS? Não dizer insightful. Estou a começar, ser mais humilde}
Malaria incidence rate, i.e., the number of reported cases per year, has been decreasing globally since 2010 \cite{who2017world}.
Fundings for disease control and prevention, although reportedly still less than required \cite{who2017world,pigott2012funding}, allowed for campaigns to include insecticide-treated mosquito nets, insecticide spraying actions, and facilitation of access to curative and preventive antimalarial drugs for millions of people at risk \cite{who2017world}.
Amongst the intervened populations, control surveys serve as an important tool, allowing to estimate malaria transmission intensity across different regions.
As malaria incidence is gradually reduced, it is required for such measuring approaches to remain as accurate as possible, since all developments made in this field require constant up to date data to keep formulating well informed actions for prevention and control \cite{who2017framework}.
% \textbf{\textcolor{red}{PREVALENCIA É UM UNIDADE ESTÁTICA, O QUE REDUZ POR UNIDADE DE TEMPO É A INCIDÊNCIA!!!!!}}
%%%%%%%%%%%%%%%%%%%%%%%%%%%%%%%%%%%%%%%%%%%%%
% ESTIMATING MALARIA INTENSITY TRANSMISSION %
%%%%%%%%%%%%%%%%%%%%%%%%%%%%%%%%%%%%%%%%%%%%%
\section{Estimating malaria transmission intensity}
\label{seq:estimating}

%%%%%%%%%%%%%%%%%%%%%%%%%%%%%%%%%%%%%%%%%%%%%
\subsection{Conventional measures to estimate transmission intensity}

Transmission intensity is the frequency with which people living in an area are bitten by the infected \textit{Anopheles} mosquitoes \cite{world2016terminology}.
% \textcolor{red}{NÃO DEFINI MALARIA TRANSMISSION INTENSITY: TENHO DE PÔR -- primeira frase do capítulo.}
Campaigns for control and elimination require knowledge and stratification of malaria transmission intensity amongst the intervened populations \cite{who2017framework, who2015guidelines}.
Based on the parasites' life cycle and its influencing determinants, several approaches can be used to estimate such transmission rate.
Currently, the most used measures for malaria transmission focus on counting the number of detectable cases of infection.
%For this thesis, the most known measures are grouped as the `classical approaches'.
%These measures usually focus on clinically detectable cases of infection, presenting similar strengths and limitations.
%For this thesis, measures used are grouped as two distinct groups.
%The first one, the `classical' approaches, identifies the most used measures.
%Those that focus on clinically detectable cases of infection.
These measures are usually based on active case detection, where infected individuals are identified by active searches for infected patients, or passive case detection, where symptomatic patients come into health facilities seeking care for their illness \cite{who2017framework,doolan2002malaria}.
In both situations, analyses are only performed on individuals presenting symptoms that evidence a possible case of infection \cite{who2017framework}.
Measures such as the parasite rate (PR) -- also known as prevalence of infection -- for assessing the proportion/prevalence of individuals with blood-stage infections in a community, or the spleen rate, for identifying the prevalence of palpable enlarged spleens due to malaria infection (an effect more commonly observed in first time infected individuals), are examples of possible population-focused approaches.
% Both methods usually assume prevalence of infection as an annual rate.
%of inference for transmission intensity uniquely based on population screening.
In order to be effective, these measures greatly depend on the diagnosis %\textcolor{red}{and treatment} 
given by established services from the public, or private health sectors, as well as coordinated community services that are the first line of action for symptom assessment and treatment, by reporting the cases to health facilities \cite{who2017framework}.
Alternatively, estimation of transmission intensity can be done by studies measuring the density of \textit{Anopheles} mosquitoes near the inspected populations.
The proportion of infected mosquitoes in a region positively correlates to the capacity of these insects to transmit malaria within that area.
This insect proportion also reflects the number of infected, and potentially infectious, human individuals \cite{who2015guidelines}.
The entomological inoculation rate (EIR) measures the number of infective mosquito bites received per person, in a population, over a defined period of time \cite{who2017framework}.
%by calculating the proportion of mosquitoes and their human biting rate \cite{warrell2002essential}, 

As interventions expand the number of populations inspected, heterogeneity in transmission intensity across different regions is likely to occur.
When assessing transmission rates across different sites, reports given by measures such as PR or EIR allow to identify determinant variables related to the parasites, the mosquitoes, or even the human hosts \cite{who2015guidelines}.
The analysis of these potential risk inducing variables can be used to define areas with high transmission rates and act accordingly \cite{who2015guidelines,world2015global}.
%Although practical and broadly used, by often depending on clinical cases, these measures can be imprecise and fail if there is a large prevalence of asymptomatic infections \cite{warrell2002essential}.

Although practical and broadly used, measures dependent on infected individuals present limitations.
% In low transmission settings, the difficulty to estimate malaria incidence intensifies.
Due to low transmission rates in certain environments, sampling infected mosquitoes and individuals can be challenging.
% \textcolor{red}{é muito difícil encontrar casos no tempo de amostragem, seja pessoas infectadas ou mosquitos infectados.}
These environments are characterised by a high number of non symptomatic cases of infection and a residual number of detectable infected mosquito bites.
As asymptomatic, the undiagnosed individuals will remain invisible to the health system while still contributing to the cycle of malaria transmission \cite{world2015global}.
Sites affected by seasonality that regularly shift between extreme high and low transmissions intensities also present a challenge to obtain accurate results \cite{cameron2015defining,o2007parasite}.
Regions where malaria incidence has been effectively reduced, or have recently been focused by campaigns, still need to be monitored in order to change interventions from malaria control and elimination to disease eradication.
For sites where malaria incidence is currently low, alternative approaches might be favourable when estimating the transmission rates \cite{corran2007serology}.
Serological antibody-based techniques can be used, as follows.
%If the study site is affected by seasonality, the results produced can also be misleading \cite{cameron2015defining}.
%The approaches present limited sensitivity to extreme high or low transmission intensities \cite{o2007parasite}.
%Isto acontece porque torna-se difícil associar os simptomas unicamente à malaria!
%In some settings the density of parasitaemia is so low in a substantial proportion of individuals that it cannot be detected with current routine diagnostic tools. These people unwittingly contribute to the cycle of malaria transmission. If future disease control and elimination strategies are to succeed, they will need to take into account this large “infectious parasite reservoir”. The expected development and availability over the next decade of new tools and approaches should help the detection and targeting of this reservoir and the clearing of plasmodia from asymptomatic carriers \cite{world2015global}.
%\textcolor{red}{geographical and demographical} 
%To deal with the  key approach to optimizing malaria responses within a country will be structuring programmes in response to stratification by malaria burden and based on an analysis of past malaria incidence data, risk determinants related to the human host, parasites, vectors and the environment that together with an analysis of access to services..
%The standard approaches are useful to estimate presence of infection in situations where endemic malaria occurs at high intensity rates, with several symptomatic cases at a time.
%It is then necessary to study possible alternatives \cite{malera2011research} in order to optimise population screenings \cite{sachs2002economic, stewart2009rapid}.
%as they tend to be expensive, time-consuming, and even with some lack of precision \cite{sachs2002economic, stewart2009rapid}.
%Assessing malaria transmission intensity and evaluating the impact of interventions is complicated in areas where transmission has been substantially reduced, as low endemic malaria often does not show evident symptoms \cite{}.
%Therefore, alternative approaches are required to assess malaria transmission and evaluate intervention programmes.


%%%%%%%%%%%%%%%%%%%%%%%%%%%%%%%%%%%%%%%%%%%%%
\subsection{Serology as an epidemiological tool}

% \textcolor{red}{Tenho de definir serology LOGO no primeiro parágrafo! Depois relacionar com os anticorpos.}
Serology-based methods inspect the densities of existing antibodies and respective antigens circulating in the serum.
Using serology, malaria transmission intensity can be assessed by identifying the levels of specific anti-malarial antibodies produced \cite{corran2007serology, drakeley2005estimating}.
Serology allows then to estimate the population level of disease transmission by appraising how a population boosts its immunity as a response to the presence or absence of infection.

% Antibodies \textcolor{red}{fazem target de tudo} are \textcolor{red}{protective--podem não ser protectores}
Antibodies are specific proteins produced by the immune system, able to recognise and target particular foreign substances, the antigen molecules.
During the course of natural infections to malaria parasites, individuals develop specific antibodies against the malarial antigens.
With multiple episodes of infection over time, a protective immunity will gradually build up, reducing manifestations of severe disease \cite{perlmann2002malaria}.
Because this process of achieving effective protection takes time, the antimalarial immunity in malaria endemic countries is said to be `age-dependent' \cite{ondigo2014estimation}.
% This gradual a\textcolor{red}{, justifying the reason for the primary risk group being children younger than 5 years old, who have yet to develop an efficient immune system \cite{snow2002consequences}. Isto é só verdade em África.}
% Pregnant woman are also a vulnerable group to malaria infections, as pregnancy affects the immune system's defences against the disease \cite{carter2002evolutionary,perlmann2002malaria}.
As the immune system reacts to the presence of malaria parasites, the identification of specific antibodies in serologic tests reflects the cumulative (age-dependent) exposure to multiple infections over time \cite{van2015serology}.
Blood samples taken at a certain time point can provide information about whether or not the individual has been infected before that time point \cite{hens2012modeling}.
This ability allows serology to function as a proxy measure of historical malaria transmission, even in low transmission settings.



When applied in epidemiological studies, serological methods shift the focus away from epidemiological measures based on infection.
% \textcolor{red}{, allowing for data sets to be based on random samples drawn from the population, while still detecting possible disease transmission heterogeneity across different epidemiological situations.}
The differentiation across multiple sites provides a better source of information than active or passive case detection that usually inspect only those who appear suspected of being infected, with possible biased results or inaccurate representative cases \cite{nkumama2017changes}.
%For that situation, a second group identifies serology as a way to measure malaria transmission intensity in situations of low and endemic malaria transmission.
%When symptomatic malaria infections occur frequently the `classical' approaches are the most commonly used measures.
%With the reduction of malaria symptomatic cases on various sites across Africa \cite{}
%Classic deterministic models
%The improvement over the last decade resulted in a 
%led to a change in malaria intervention objectives, shifting from controlling towards elimination \cite{kitua2011conquering}.
%Serology is based on the human anti-malarial antibodies gain by individuals exposed to malaria parasites.
% \\
% \textcolor{red}{In normal conditions}, an exposed healthy individual will gradually develop and accumulate the anti-malarial antibodies.
% This effect can be better seen in older children and adults, where the frequency and severity of the disease are reduced when compared to younger individuals \cite{snow2002consequences}.
% \textcolor{red}{Blood samples are usually taken, producing... SER MAIS PRECISO} 
Serology has increasingly been incorporated in cross-sectional and longitudinal studies to monitor recent population changes in transmission intensity \cite{cook2010using, cook2011serological, hay2008measuring} and evaluate effectiveness of malaria eradication efforts \cite{bruce1973seroepidemiological}.
% , identify hotspots for transmission

The antibodies produced upon exposure to malaria parasites belong to the acquired immune system.
The specific antimalarial antibodies, contrarily to some diseases such as some forms of the hepatitis virus, the mumps, or the rubella virus, wane over time in absence of infection.
This means malaria does not cause long-lasting immunity.
After a prolonged interval without reinfection, immunologically protected individuals can revert to an naïve status and once again become vulnerable to show symptoms.
%These specific protective antibodies are thought to have some effect on the transmission dynamics of malaria . Loss of protective immunity
However, in malaria endemic sites, individuals might be exposed to a somewhat constant rate of infection from an early age.
In those scenarios, the acquired immunity, i.e., the gradual learning of the immune system upon multiple exposures, grants antibody persistence due to continuous exposure over a long period time.
In cases of endemic malaria, data from a single cross-sectional survey can be used to generate a point estimate of the current disease transmission intensity.
The measure can also analyse potential historical changes in transmission intensity that led to a variation in exposure to the parasite \cite{hens2012modeling}.
%Because serological markers provide information on cumulative exposure over time they are particularly well suited for evaluating long-term transmission trends \cite{corran2007serology,drakeley2005estimating}. Data from a single cross-sectional serological survey can be used
%With the recent progress made in reducing the global malaria burden, alternative approaches that correctly identify asymptotic infections are becoming fundamental \cite{malera2011research}.
%As a proxy measure of malaria transmission, serological responses to P. falciparum antigens have shown a robust and consistent correlation with estimates of EIR \cite{corran2007serology}, and thus 
%Immunity in malaria does not fully protect against infection of disease and it may have poor immune memory \cite{struik2004does}.
%Thus, herd immunity does not tend to develop.
%In highly endemic areas, immunity reduces parasite densities in older children and adults and it reduces the frequency and severity of the disease 
%challenges for malaria em low transmission settings \cite{stresman2012malaria} principal serologic markers to detect malaria em low transmission settings \cite{bousema2010serologic} (pfMSP-1 and pfAMA-1)
%Under stable endemic malaria conditions, variation in transmission can continue even with very few vectors.
%High levels of immunity develop within the population due to regular and often continuous transmission \cite{warrell2002essential}.
%Effective antiparasitic immunity is achieved only after multiple infections
%The advantage of serological data in quantifying parasite exposure instead of infection has recently brought interest in sero-epidemiological studies of malaria \cite{}.
%Alternative method is to examine the prevalence of some markers of previous infection that is present in blood serum -- a measure of the proportion of humans currently with antibodies that developed in response to malaria infection \cite{corran2007serology}.
%A unique attribute of antibody measurements is that they provide an immunological record of an individual's exposure or vaccination history, and thus integrate information over time \cite{corran2007serology}.
%Typically seroprevalence rises with age and gives a robust measure of previous infection compared with age.
%Introduction to serology
%Individuals are born seronegative but can be converted into seropositive upon malaria exposure. In the absence of continuous frequent malaria exposure, these individuals can then revert to a seronegative immunological state.
%Explain about the relation between classical EIR and the measure of transmission intensity on serology (SCR).
%Explicar o processo de 'ser exposto' e passar ao estado infectado, e depois recuperar. Falar sobre o processo de desenvolver anticorpos específicos no processo de combater a infeção.
%\cite{arnold2017measuring}


%%%%%%%%%%%%%%%%%%%
% BENCHMARK STUDY %
%%%%%%%%%%%%%%%%%%%
\section{Northeast Tanzania as a serological benchmark study}
\label{seq:example}

%Serology has been gaining importance as it has proven to be effective when measuring transmission intensity in situations of low and endemic malaria transmission.
Serology and sero-epidemiological studies have already been assessed as good alternatives to analyse situations of malaria in stages of pre-elimination and elimination \cite{corran2007serology}.
A benchmark example is the study whose data set is used throughout this thesis \cite{drakeley2005altitude}.
The study in Northeast Tanzania applied serology-based methods to analyse cohorts of patients from 24 distinct villages with varying intensities of transmission.
Inferences made about the sites' seroprevalence  -- measure for the proportion of \textit{P. falciparum}-specific antigen seropositive individuals detected in each community -- allowed researchers to describe malaria transmission intensity from the different villages as function of altitude and estimated rainfall, confirming both variables to have a measurable impact on the disease's force of infection, i.e., the rate at which non immune, susceptible individuals become infected.

Following the described study, several published articles used and improved the sero-epidemiological methodologies and inferences.
Based on the same data set alone, studies of methods and approaches in various research fields were developed.
Some examples are studies on the serological analyses, inquiring about the trends in malaria endemicity \cite{drakeley2005estimating}, genetic studies on populations exposed to \textit{P. falciparum} parasites \cite{enevold2007associations, sepulveda2017malaria}, and development of specific mathematical model, used for serological analyses \cite{bosomprah2014mathematical}.
%Due to the variety and quality of information gathered, this data set has been used on multiple different studies with the focus being serological analysis, inquiring about the trends in malaria endemicity \cite{drakeley2005estimating}; genetic studies on populations exposed to \textit{P. falciparum} parasite \cite{enevold2007associations,sepulveda2017malaria}; and development of specific mathematical model, used for serological analysis \cite{bosomprah2014mathematical}.


%%%%%%%%%%%%%%%%%%%%%%
% CURRENT CHALLENGES %
%%%%%%%%%%%%%%%%%%%%%%
\section{Current challenges on malaria epidemiology}
\label{seq:challenges}

% \textcolor{red}{Esta section não faz muito sentido aqui}
The multidisciplinary investment to control and aid populations hurt by the endemic malaria burden is visible \cite{who2017world}.
Nowadays, severe malaria develops only in a minority of sites as effective campaigns have been able to control and reduce disease transmission intensity substantially \cite{marsh1995indicators}.
Low transmission settings are now registered across various regions \cite{cook2010using}.
%More regions are reducing malaria to low transmission settings with campaigns, switching their focus from sustained control to elimination \cite{cook2010using}.
%All efforts have resulted in areas of low transmission intensity,
%or areas where transmission has been reduced substantially, 
%malaria control programmes can start considering switching from sustained control initiatives to elimination \cite{cook2010using}.
%The  measure the fraction of the population with particular conditions at some point in time.
All measures presented here estimate malaria transmission intensity on the human population.
However, none of them is a perfect indicator.
%Treated infections can clear rapidly and even untreated infections can also clear after some time, so the classical approaches are but measures of recent infection.
Prevalence of malaria presents a characteristic pattern of increasing with age in young children under five years old, only to then decline throughout adolescence and adulthood (as individuals develop protective immunity).
This age-dependent fluctuation is defined as `peak-shift' and can be difficult to estimate, making approaches such as PR or EIR poor indexes of transmission intensity over time.
These measures can be used as good alternatives to estimate recent infections.
%Clinical incidence also displays age-specific patterns that can differ by endemicity, making the measures unreliable indexes of transmission intensity, as the screened regions may present heterogeneity in malaria transmission intensity.
For serological analyses, some infections may be treated and clear before an immune response develops.
% In other scenarios, individual characteristics, such as genetics, can also present a significant effect in the immune response.
This possible lack of immunity development, as well as waning immune responses from the malarial antibodies can affect the accuracy of seroprevalence.

%%%%%%%%%%%%%%%%%%%%%%%%
% OBJECTIVES & OUTLINE %
%%%%%%%%%%%%%%%%%%%%%%%%
\section{Objectives and outline}
\label{seq:objectives}

% This thesis will focus on the effects of the parasitic species \textit{P. falciparum}.

Three \textit{P. falciparum}-specific antigens were measured and analysed in order to estimate malaria transmission intensity: the merozoite surface protein 1 (MSP1), the merozoite surface protein 2 (MSP2), and the apical membrane antigen 1 (AMA1).
The corresponding antibodies are known to not confer effective protection against malaria.
Instead, they are used as serological markers due to their immunogenic profile, meaning they are expected to be detected and indicate exposure to malaria even in low transmission intensity settings \cite{reddy2012high, wong2014serological, bousema2010serologic}.
% MSP1 is the antigen most extensively characterised and has been implicated as a target for protective immune responses in a large number of studies \cite{perlmann2002malaria}.

To analyse the Tanzania data set, different statistical approaches were applied.
Using infection and serological samples from the different \textit{P. falciparum} antigens, one expects to identify the principal determinants influencing the prevalence of infection, as well as estimate seroconversion rate -- average rate at which individuals become positive for the antimalarial antibodies, upon exposure to \textit{P. falciparum} parasites.
%identify the expected heterogeneity amongst different sites through the use of different statistical approaches that relate to a real life scenario.
Based on the aphorism that all models are wrong but some are useful \cite{box2005statistics}, different statistical models were fit to the data.
First, by making use of the generalised linear models and more commonly used statistical approaches to study the detected infection cases and prevalence of infection.
Afterwards, the seroprevalence was studied by applying specific serological models to the different known antigens.
%Both approaches aim to describe the disease heterogeneity present across all populations studied.
%both introduced approaches, with aim to describe the disease heterogeneity present across all populations studied.
%The prevalence and seroprevalence values are used, measuring current malaria transmission intensity based on past exposure or past interventions.
%Estimate and measure malaria transmission intensity by

% \textcolor{red}{Isto é converas de café:}
As a thesis in Biostatistics, this project was structured in a way that would focus different academic objectives, while maintaining a coherent line of thought.
With its fundaments in the matters of malaria and malaria sero-epidemiology, this project granted the opportunity to:
\begin{itemize}
\item Work with a well recorded cross-sectional data, used in renowned sero-epidemiological studies;
\item Build different generalised linear models to characterise the risk factors and study effects causing heterogeneity in transmission intensity levels;
\item Make use of the specific reverse catalytic models to corroborate the previous heterogeneity inferences, and create serological profiles for different villages based on different biological and epidemiological assumptions;
\item Describe and propose an innovative extension of the more broadly used reverse catalytic models;
\item Study the implications of applying different models to the same data set by changing between epidemiological strategies.
\end{itemize}
%When performing inference for serology, two of the three models here described are known approaches, with one of the models here proposed being innovative.
%\textbf{
%objective: ja está feito: SCR constante ao longo do tempo\\ changes in SCR and changes in SRR e ver se as condições mudam com o que já está reportado.}
%The main objective of the thesis is to estimate and measure malaria transmission intensity from a set of different groups of villages, identify the expected heterogeneity amongst different sites through the use of different statistical approaches that relate to a real life scenario.
%Based on the aphorism that all models are wrong but some are useful \cite{}, different models are created and compared.
%The prevalence and seroprevalence values are used, measuring current malaria transmission intensity based on past exposure or past interventions.
%The thinking that was made during the creation of the project.\\
%The analysis that were made, the models created and the order in which they were created. The creation of the so called Infection Model.\\
%Etc.
%George P. Box -- \emph{´All models are wrong, but some are useful.'} -- The practical question is to how wrong do they have to be to no be useful?
%With this thesis different statistical approaches are described and used onto the same data serological set.
%All different models created aim to describe the disease heterogeneity, present across all populations studied.

The following chapter will specify the situation of malaria in the two studied regions of the Northeast Tanzania, as well as introduce, and describe the collected data (Chapter \ref{ch:2.0}).
The statistical theory used is described further ahead (Chapter \ref{ch:3.0}).
The models' analyses and inferences are then presented, firstly focusing the infection status of each individual as the outcome of interest.
Afterwards, by applying the alternative statistical methodologies onto the serological outcomes (Chapters \ref{ch:4.0} and \ref{ch:5.0}, respectively).
Lastly, the different methodologies and results are discussed, attending the statistical and epidemiological backgrounds of this project (Chapter \ref{ch:discussion}).