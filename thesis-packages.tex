% ADDED PACKAGES AND STRUCTURE OPTIONS ------------------------------

% \input{uhead.sty}
% \setlength{\parskip}{12pt}        % a little space before a \par
\setlength{\parindent}{0pt}	      % don't indent first lines of paragraphs % identation?
% \input{boxit.sty}   % to draw a box around the text - labels the page with the chapters
\makeatletter  %to avoid error messages generated by "\@". Makes Latex treat "@" like a letter


%%%%%%%%%% What do I need to fill %%%%%%%%%%
\def\university#1{\gdef\@university{#1}}
\def\faculty#1{\gdef\@faculty{#1}}
\def\dept#1{\gdef\@dept{#1}}
\def\title#1{\gdef\@title{#1}}
\def\masters#1{\gdef\@masters{#1}}
\def\supervisor#1{\gdef\@supervisor{#1}}
\def\supervisora#1{\gdef\@supervisora{#1}}
\def\submitdate#1{\gdef\@submitdate{#1}}
%%%%%%%%%%%%%%%%%%%%%%%%%%%%%%%%%%%%%%%%%%%%

%%%%%%%%%%%%%%%%%%%%%%
%%%%% TITLE PAGE %%%%%
%%%%%%%%%%%%%%%%%%%%%%
\def\maketitle{
    \begin{titlepage}
    \renewcommand{\baselinestretch}{1.5}
    \fontfamily{phv}\fontseries{mc}\selectfont
	\begin{center} \thispagestyle{empty}  
	\par{\LARGE \@university}\\
        \par{\LARGE \@faculty}\\
        \par{\LARGE \@dept}\\
	\vfill
	
	\includegraphics[scale=0.25]{images/logo_fcul.pdf}
	
	\vfill
	\LARGE \textbf{Statistical approaches to measure heterogeneity in malaria transmission intensity:\\ An epidemiological study on the Tanzania populations}
	\vfill
	\par{\LARGE \@author}\\
	\vfill
	\large \textbf{Mestrado em Bioestatística} \\
	\vspace{1.2cm}
% 	\large Versão Provisória \\
	\vspace{0.8cm}
	\large Dissertação orientada por: \\
	%\vspace{1cm}
	\large Prof. Doutor Nuno Sepúlveda\\
	\large Prof.ª Doutora Marília Antunes\\
	\vfill
	\par{\large \@submitdate}
	\end{center}
\end{titlepage}
}

% \def\maketitle{
%   \begin{titlepage}
%   \thispagestyle{empty}
%     \begin{center}
        
%         \par{\LARGE \@university}\\
%         \par{\LARGE \@faculty}\\
%         \par{\LARGE \@dept}\\
        
%         \vspace{1.0cm}

% 		\includegraphics[scale=0.25]{images/logo_fcul.pdf}\\
 
% 		\vspace{0.5cm}
        
%         \Huge
% 		\textbf{\@title}
        
%         \vspace{1.0cm}
        
%         \par{\LARGE \@author}\\
%         \vspace{1.0cm}
%         \Large
%         \textbf{\@masters}
        
%         \vspace{1.0cm}
%         \large
%         \textbf{(Versão provisória)}\\
%         Dissertação orientada por:\\\@supervisor\\\@supervisora

        
%         \vspace{1.0cm}
        
%         \par{\Large \@submitdate}
		
        

        
%     \end{center}
% \end{titlepage}
% }


\def\titlepage{
  \newpage
  \centering
  \linespread{1.15}
  \normalsize
  \vbox to \vsize\bgroup\vbox to 9in\bgroup
}

\def\endtitlepage{
  \par
  \kern 0pt
  \egroup
  \vss
  \egroup
  \cleardoublepage
}
%%%%%%%%%%%%%%%%%%
%%%%% RESUMO %%%%%
%%%%%%%%%%%%%%%%%%
\def\resumo{
  \begin{center}{
    \large\bf Resumo}
  \end{center}
  \small
%   \def\baselinestretch{1.15}
  \linespread{1.15}
  \normalsize
}
\def\endresumo{
  \par
}
%%%%%%%%%%%%%%%%%%%%
%%%%% ABSTRACT %%%%%
%%%%%%%%%%%%%%%%%%%%
\def\abstract{
  \begin{center}{
    \large\bf Abstract}
  \end{center}
  \small
%   \def\baselinestretch{1.15}
  \linespread{1.15}
  \normalsize
}
\def\endabstract{
  \par
}
%%%%%%%%%%%%%%%%%%%%%%%%%%%%
%%%%% ACKNOWLEDGEMENTS %%%%%
%%%%%%%%%%%%%%%%%%%%%%%%%%%%
\newenvironment{acknowledgements}{
  \cleardoublepage
  \begin{center}{
    \large \bf Acknowledgements}
  \end{center}
  \small
  \linespread{1.15}
  \normalsize
}{\cleardoublepage}
\def\endacknowledgements{
  \par
}
%%%%%%%%%%%%%%%%%%%%%%
%%%%% DEDICATION %%%%%
%%%%%%%%%%%%%%%%%%%%%%
\newenvironment{dedication}{
  \cleardoublepage
  \begin{center}{
    \large \bf Dedication}
  \end{center}
  \small
  \linespread{1.15}
  \normalsize
}{\cleardoublepage}
\def\enddedication{
  \par
}
%%%%%%%%%%%%%%%%%%%
%%%%% PREFACE %%%%%
%%%%%%%%%%%%%%%%%%%
\def\preface{
    \pagenumbering{roman}
    \pagestyle{plain}
    \doublespacing
}
%%%%%%%%%%%%%%%%
%%%%% BODY %%%%%
%%%%%%%%%%%%%%%%
\def\body{
    \cleardoublepage
    \makeatletter
    % \pagestyle{headings}
    \tableofcontents
    \pagestyle{plain}
    \cleardoublepage
    
    % \pagestyle{headings}
    \listoftables
    \pagestyle{plain}
    \cleardoublepage
    
    % \pagestyle{headings}
    \listoffigures
    \pagestyle{plain}
    \cleardoublepage
    
    % \pagestyle{headings}
    \pagenumbering{arabic}
    \linespread{1.15}
}
\makeatother  %to avoid error messages generated by "\@". Makes Latex treat "@" like a letter

% %%%%%%%%%%%%%%%%
% %%%%% BODY %%%%%
% %%%%%%%%%%%%%%%%
% \def\body{
%     %%% Contents %%%
%     % \cleardoublepage    
%     \pagestyle{uheadings}
%     \tableofcontents
%     \pagestyle{plain}
    
%     %%% Tables %%%
%     %\cleardoublepage
%     \pagestyle{uheadings}
%     \listoffigures
%     \pagestyle{plain}
%     %\cleardoublepage
    
%     %%% Figures %%%
%     \pagestyle{uheadings}
%     \listoftables
%     \pagestyle{plain}
%     %\cleardoublepage
    
%     %%% Chapters %%%
%     \pagestyle{uheadings}
% }

% \makeatother  %to avoid error messages generated by "\@". Makes Latex treat "@" like a letter % knit thesis structure


%%%%%%%%%%%%%%%%%%%%
%%%%% PACKAGES %%%%%
%%%%%%%%%%%%%%%%%%%%

\usepackage{adjustbox}
\usepackage{afterpage}
\usepackage{amsfonts}
\usepackage{amsmath} % for intense use of mathematical symbols
\usepackage{amssymb} % forarrows and such in equations
\usepackage{amsthm}
\usepackage{array}
\usepackage{auto-pst-pdf}
\usepackage[toc,page]{appendix}
\usepackage{bm} % bold symbols
\usepackage{booktabs}
\usepackage{breqn} % automatic line breaking in equations - useful
\usepackage[font=small, labelfont=bf]{caption}
\usepackage{color}
\usepackage{enumitem}
\usepackage{epsf}
\usepackage{epsfig}
\usepackage{epstopdf}
\usepackage{fancybox}
\usepackage{float} % figures and tables
\usepackage{geometry}
\usepackage{graphicx} % import graphics
\usepackage[hidelinks]{hyperref}  % for references and labels
\usepackage{latexsym} % some extra symbols
\usepackage{listings}
\usepackage{mathchars} % presentation of math characters
\usepackage{makecell}
\usepackage{multirow}
\usepackage[numbers,sort]{natbib} % citations
\usepackage{paralist}    % To enable customise enumerates
\usepackage{pst-plot}
\usepackage{pstricks}
\usepackage{pstricks-add}
\usepackage{rotating}
\usepackage{scrextend} % add lists
\usepackage{setspace} % set spacing between the lines
\usepackage{soul}
\usepackage{subcaption} % for subfigures
\usepackage{tabularx}
\usepackage{tikz}
\usepackage{tikzinput}
\usepackage[colorinlistoftodos, textwidth=60, textsize=small]{todonotes}
\usepackage[normalem]{ulem} % underline options
\usepackage{upgreek}
\usepackage{url} % allows linebreaks at certain characters or combinations of characters
\usepackage{verbatim} % comment environment
% \usepackage[svgnames]{xcolor}



%%%%%%%%%%%%%%%%%%%%%%%%%%%%%%%%
%%%%% EXTRAS FROM TEMPLATE %%%%%
%%%%%%%%%%%%%%%%%%%%%%%%%%%%%%%%

\let\origappendix\appendix % save the existing appendix command
\renewcommand\appendix{\clearpage\pagenumbering{roman}\origappendix}

\def\undertilde#1{\mathord{\vtop{\ialign{##\crcr $\hfil\displaystyle{#1}\hfil$\crcr\noalign{\kern1.5pt\nointerlineskip}
$\hfil\tilde{}\hfil$\crcr\noalign{\kern1.5pt}}}}}

\def\coop{\mbox{\large $\rhd\!\!\!\lhd$}}
\newcommand{\sync}[1]{\raisebox{-1.0ex}{$\;\stackrel{\coop}{\scriptscriptstyle
#1}\,$}}


%%%%%%%%%%%%%%%%%%%%%%%%
%%%%% NEW COMMANDS %%%%%
%%%%%%%%%%%%%%%%%%%%%%%%
\newcommand{\ipc}{{\sf ipc}}
\newcommand{\Prob}{\bbbp}
\newcommand{\Real}{\bbbr}
\newcommand{\real}{\Real}
\newcommand{\Int}{\bbbz}
\newcommand{\Nat}{\bbbn}
% syze of letters ------------------------------
\newcommand{\NN}{{\sf I\kern-0.14emN}}   % Natural numbers
\newcommand{\ZZ}{{\sf Z\kern-0.45emZ}}   % Integers
\newcommand{\QQQ}{{\sf C\kern-0.48emQ}}   % Rational numbers
\newcommand{\RR}{{\sf I\kern-0.14emR}}   % Real numbers
% mathematical letters ------------------------------
\newcommand{\KK}{{\cal K}}
\newcommand{\OO}{{\cal O}}
% bold letters ------------------------------
\newcommand{\AAA}{{\bf A}}
\newcommand{\HH}{{\bf H}}
\newcommand{\II}{{\bf I}}
\newcommand{\LL}{{\bf L}}
\newcommand{\PP}{{\bf P}}
\newcommand{\PPprime}{{\bf P'}}
\newcommand{\QQ}{{\bf Q}}
\newcommand{\UU}{{\bf U}}
\newcommand{\UUprime}{{\bf U'}}
\newcommand{\zzero}{{\bf 0}}
\newcommand{\ppi}{\mbox{\boldmath $\pi$}}
\newcommand{\aalph}{\mbox{\boldmath $\alpha$}}
\newcommand{\bb}{{\bf b}}
\newcommand{\ee}{{\bf e}}
\newcommand{\mmu}{\mbox{\boldmath $\mu$}}
\newcommand{\vv}{{\bf v}}
\newcommand{\xx}{{\bf x}}
\newcommand{\yy}{{\bf y}}
\newcommand{\zz}{{\bf z}}
\newcommand{\oomeg}{\mbox{\boldmath $\omega$}}
\newcommand{\res}{{\bf res}}
\newcommand{\cchi}{{\mbox{\raisebox{.4ex}{$\chi$}}}}
%\newcommand{\cchi}{{\cal X}}
%\newcommand{\cchi}{\mbox{\Large $\chi$}}
% Logical operators and symbols ------------------------------
\newcommand{\imply}{\Rightarrow}
\newcommand{\bimply}{\Leftrightarrow}
\newcommand{\union}{\cup}
\newcommand{\intersect}{\cap}
\newcommand{\boolor}{\vee}
\newcommand{\booland}{\wedge}
\newcommand{\boolimply}{\imply}
\newcommand{\boolbimply}{\bimply}
\newcommand{\boolnot}{\neg}
\newcommand{\boolsat}{\!\models}
\newcommand{\boolnsat}{\!\not\models}
\newcommand{\op}[1]{\mathrm{#1}}
\newcommand{\s}[1]{\ensuremath{\mathcal #1}}
% Properly styled differentiation and integration operators ------------------------------
\newcommand{\diff}[1]{\mathrm{\frac{d}{d\mathit{#1}}}}
\newcommand{\diffII}[1]{\mathrm{\frac{d^2}{d\mathit{#1}^2}}}
\newcommand{\intg}[4]{\int_{#3}^{#4} #1 \, \mathrm{d}#2}
\newcommand{\intgd}[4]{\int\!\!\!\!\int_{#4} #1 \, \mathrm{d}#2 \, \mathrm{d}#3}
% Large () brackets on different lines of an eqnarray environment ------------------------------
\newcommand{\Leftbrace}[1]{\left(\raisebox{0mm}[#1][#1]{}\right.}
\newcommand{\Rightbrace}[1]{\left.\raisebox{0mm}[#1][#1]{}\right)}
% Funky symbols for footnotes ------------------------------
\newcommand{\symbolfootnote}{\renewcommand{\thefootnote}{\fnsymbol{footnote}}}
% now add \symbolfootnote to the beginning of the document...
\newcommand{\normallinespacing}{\renewcommand{\baselinestretch}{1.15} \normalsize}
\newcommand{\mediumlinespacing}{\renewcommand{\baselinestretch}{1.2} \normalsize}
\newcommand{\narrowlinespacing}{\renewcommand{\baselinestretch}{1.0} \normalsize}
\newcommand{\bump}{\noalign{\vspace*{\doublerulesep}}}
\newcommand{\cell}{\multicolumn{1}{}{}}
\newcommand{\spann}{\mbox{span}}
\newcommand{\diagg}{\mbox{diag}}
\newcommand{\modd}{\mbox{mod}}
\newcommand{\minn}{\mbox{min}}
\newcommand{\andd}{\mbox{and}}
\newcommand{\forr}{\mbox{for}}
\newcommand{\EE}{\mbox{E}}
\newcommand{\deff}{\stackrel{\mathrm{def}}{=}}
\newcommand{\syncc}{~\stackrel{\textstyle \rhd\kern-0.57em\lhd}{\scriptstyle L}~}


\newtheorem{definition}{Definition}[chapter]
\newtheorem{theorem}{Theorem}[chapter]