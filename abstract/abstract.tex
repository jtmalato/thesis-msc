
\addcontentsline{toc}{chapter}{Abstract}

\begin{abstract}

% To this day, malaria continues to be a worldwide cause of death and disease.
% With the recent decades bringing insightful research studies, campaigns for control and elimination became more efficient, gradually reducing the \textit{Plasmodium falciparum} parasitic malaria across sub-Saharan African countries.
% Such actions have resulted in regions of pre-elimination going into elimination stages, where detectable symptomatic infections are almost vestigial.
% These scenarios may impose a new challenge, as the more usual methodologies do not consider apparent invisible individuals when estimating malaria transmission intensity and its prevalence of infection.
% As a proposed alternative to this question, sero-epidemiology can be used to more accurately perform such inferences.

% The objective of this project is then to estimate and characterise the \textit{P. falciparum} transmission intensity from a sample of 5058 individuals structured by age groups, from across 21 villages in the Northeast Tanzania, with different prevalence levels.
% First, the principal transmission determinants influencing the infection heterogeneity were identified.
% Using the generalised linear models (GLMs) the study revealed the importance of some demographical risk factors when assessing the presence/absence of infection.
% Determinants such as the age group of the individuals, the altitude of a village -- a known proxy of transmission intensity --, or the transect in which the villages are encompassed, were some of the more impactful demographical transmission determinants assessed.
% The detected antimalarial antibodies for the specific antigens MSP1, MSP2, and AMA1, used throughout this thesis, were also included in the GLMs and showed the importance these exposure determinants have as reasonable indicators of malaria infection.
% The inference then led to the use of different reverse catalytic models (RCMs), applied solely to the serological data sets of the three antigens collected. 

% The RCMs assume that individuals transit between two possible serological states (seronegative and seropositive) at distinct rates: seroconversion rate (SCR) and seroreversion rate (SRR).
% The SCR is the annual average rate by which the individuals of a certain age change from seronegative to seropositive, upon malaria infection.
% And the SRR is the annual average rate by which seropositive individuals of a certain age return to the seronegative state due to antibody decay.
% Focusing on different biological and epidemiological proposals -- that might present an effect on the annual transitional rates between seronegative and seropositive individuals due to parasite exposure -- four RCMs were tested.
% Model M$_0$ assumed the seroconversion rate (SCR) and the seroreversion rate (SRR) as constant transition values across all ages.
% Models M$_{1,1}$ and M$_{1,2}$ were built to adjust for the biological effect of gradually developing immunity over time, in a scenario of endemic malaria transmission.
% Both models assumed SCR as constant over time, with SRR being reduced to a second rate given an estimated cutoff.
% Model M$_{1,2}$, a more restrictive version, assumed that the reduced SRR would be equal to zero, with no seropositive individuals transiting into a seronegative state after the age cutoff.
% Finally, model M$_{2}$ proposed the epidemiological effect that some event (e.g.: possible campaigns or interventions to prevent malaria) might have had on transmission intensity.
% This model assumed SRR as constant rate across all ages, with a change in SCR happening under an estimated cutoff sometime before the sampling. 

% The results showed that, depending on the antigen, the models could be used to describe the transmission intensity and seroprevalence of the assessed villages.
% However, the traditional RCM M$_{0}$ (transitional rates constant over time) was more often preferred when compared to the age-dependent M$_{1,2}$ (likelihood ratio test, p-values $>0.05$).
% The age-dependent SRR model was only significant when applied to some villages at intermediate altitudes (600 meters to 1200 meters high).
% The traditional model was also chosen in favour of the model admitting a past change in transmission intensity, M$_{2}$, with the epidemiological model only identifying a change in few villages placed at low and intermediate altitudes.
% Despite the limited information to estimate some of the models' parameters, further analyses demonstrated that the statistically and overall more parsimonious model M$_0$ produced underestimations in its transitional rates, when compared to the more realistic model M$_{1,2}$.
% This underestimation could have a negative impact when estimating malaria transmission intensity in low transmission scenarios.
% Sided with the newly formulated strategies to advance sites in stages of malaria elimination and pre-elimination into eradication, serology serves as tool to more efficaciously measure transmission intensity.
% The improvement and application of the RCMs M$_{1,1}$ and M$_{1,2}$ could bring more information to the importance of a more precise estimation of the SRR.
To this day, malaria continues to be a worldwide cause of death and disease.
With the recent decades bringing insightful research studies, campaigns for control and elimination became more efficient, gradually reducing the \textit{Plasmodium falciparum} parasitic malaria across sub-Saharan African countries.
Such actions have resulted in regions of pre-elimination going into elimination stages, where detectable symptomatic infections are almost vestigial.
These scenarios may impose a new challenge, as the more usual methodologies do not consider apparent invisible individuals when estimating malaria transmission intensity and prevalence of infection.
As a proposed alternative to this question, sero-epidemiology can be used to more accurately perform such inferences.

The objective of this thesis is then to estimate and characterise the \textit{P. falciparum} transmission intensity from a sample of 5058 individuals structured by age groups, from across 21 villages in the Northeast Tanzania, with different prevalence levels.
First, the principal transmission determinants influencing the infection heterogeneity were identified.
Using the generalised linear models (GLMs) the study revealed the importance of some demographical risk factors when assessing the presence/absence of infection.
Determinants such as the age group of the individuals, the altitude of a village -- a known proxy of transmission intensity --, or the transect in which the villages were encompassed, were some of the more impactful demographical transmission determinants assessed.
The detected antimalarial antibodies for the specific antigens MSP1, MSP2, and AMA1, used throughout this thesis, were also included in the GLMs and showed the importance these exposure determinants have as reasonable indicators of malaria infection.
The inference then led to the use of different reverse catalytic models (RCMs), applied solely to the serological data sets of the three antigens collected.

The RCMs assume that individuals transit between two possible serological states (seronegative and seropositive) at distinct rates: seroconversion rate (SCR) and seroreversion rate (SRR).
The SCR is the annual average rate by which the individuals of a certain age change from seronegative to seropositive, upon malaria infection.
And the SRR is the annual average rate by which seropositive individuals of a certain age return to the seronegative state due to antibody decay.
Focusing on different biological and epidemiological proposals -- that might present an effect on the annual transitional rates between seronegative and seropositive individuals due to parasite exposure -- four RCMs were tested.
Model M$_{0}$ assumed the seroconversion rate (SCR) and the seroreversion rate (SRR) as constant transition values across all ages.
Models M$_{1,1}$ and M$_{1,2}$ were built to adjust for the biological effect of gradually developing immunity over time, in a scenario of endemic malaria transmission.
Both models assumed SCR as constant over time, with SRR being reduced to a second rate given an estimated cutoff.
Model M$_{1,2}$, a more restrictive version, assumed that the reduced SRR would be equal to zero, with no seropositive individuals transiting into a seronegative state after the age cutoff.
Finally, model M$_{2}$ proposed the epidemiological effect that some event (e.g.: possible campaigns or interventions to prevent malaria) might have had on transmission intensity. 
This model assumed SRR as constant rate across all ages, with a change in SCR happening under an estimated cutoff sometime before the sampling.

The results showed that, depending on the antigen, the models could be used to describe the transmission intensity and seroprevalence of the assessed villages.
However, the traditional RCM M$_{0}$ (transitional rates constant over time) was more often preferred when compared to the age-dependent M$_{1,2}$ (likelihood ratio test, p-values $>0.05$). The age-dependent SRR model was only significant when applied to some villages at intermediate altitudes (600 meters to 1200 meters high). The traditional model was also chosen in favour of the model admitting a past change in transmission intensity, M$_{2}$, with the epidemiological model only identifying a change in few villages placed at low and intermediate altitudes.
Despite the limited information to estimate some of the models' parameters, further analyses demonstrated that the statistically and overall more parsimonious model M$_{0}$ produced underestimations in its transitional rates, when compared to the more realistic model M$_{1,2}$.
This underestimation could have a negative impact when estimating malaria transmission intensity in low transmission scenarios.
Sided with the newly formulated strategies to advance sites in stages of malaria elimination and pre-elimination into eradication, serology serves as tool to more efficaciously measure transmission intensity.
The improvement and application of the RCMs M$_{1,1}$ and M$_{1,2}$ could bring more information to the importance of a more precise estimation of the SRR.

\textbf{Keywords:} malaria, transmission intensity, transmission heterogeneity, epidemiology, serological data, seroprevalence, seroconversion rate, seroreversion rate.


\end{abstract}
